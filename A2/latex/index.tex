\begin{DoxyAuthor}{Author}
Kai Brusch 

Matthias Nitsche 

Swaneet Kumar 

Ivan Morozov
\end{DoxyAuthor}
Protokoll – Und was wir gelernt haben\-: • Wir hatten bereits auf einem Mac mit X\-Code(einer I\-D\-E für die C Sprache) einen Tielder Aufgabe bearbeitet. Dort waren alle Funktionen enthalten, bis auf die beiden get\-\_\-sticks und put\-\_\-sticks sowie dem disp\-\_\-philo\-\_\-states. Bei der Bewältigung der Aufgabe haben wir uns öfters an dem vergegebenen Beispiel critsect02.\-c sowie an anderen Codebeispielen aus dem pub orientiert. Wir hatten auch Hinweise von der anderen Partnergruppe die fertig war erhalten. Wir haben uns an unserer Vorlage aus X\-Code orientiert und nach und nach Bestandteile unseres Codes auf der Open\-Suse-\/\-V\-M importiert und getestet. Wir haben zunächst die Initialisierung und den Lebenyzyklus der Threads/\-Philosophen sowie den Input\-Loop mit den Benutzerbefehlen eingefügt. Nachdem wir die Ausgabe der Philosophen(disp\-\_\-philo\-\_\-states) geschrieben hatten, haben wir uns an dem get\-\_\-sticks und put\-\_\-sticks rangesetzt und unter Verwendung der genannten Threadfunktionen eine funktionierende Simulation der Spesenden Philosophen erstellt. • Unser makefile enthält auch ein clean und all was den Vorgang des Compilens erleichtert. • Wir haben den gcc-\/\-Compiler im makefile auch den Befehl -\/std=c99 gegeben, damit wir Iterationsvariablen direkt im Kopf der For-\/\-Schleifen definieren können und diese daher lokal halten können. Das Default-\/\-Verhalten lässt diese (für uns als A\-I-\/ler angewohnte) Deklaration nicht zu. • Es ist nicht klug auf einem Mac mit X\-Code zu entwickeln, da viele Sachen auf der V\-M nicht mehr funktionierten. Nächstes mal, werden wir direkt auf open\-Suse in der V\-M schreiben und auf der V\-M schrittweise testend entwickeln. • Nicht nur haben wir viel über Threads und Synchronisierung in C verstanden, sondern auch über die Organisation einer größeren Gruppe – und dass größere Gruppe schwerer zu koordinieren sind und nicht unbedingt (für diesen Aufgabenkontext) einen Nutzen gebracht haben. • Wir haben auch festgestellt, dass das Git-\/\-Protokoll zumsammen mit Github sehr nützlich als Versionsmanagement für die Gruppenarbeit und die Verteilung war. • Dennoch heißt das nicht, dass jeder der einen Notepad++ hat, auch unbedingt Änderungen am Code machen und sie (auf Git\-Hub) pushen sollte. Ohne I\-D\-E Änderungen zu machen ruft viele Fehler hervor, die nicht vom Änderdernden sofort festgestellt werden können. Dies e müssen dann mühselig von Anderen behoben werden. „\-Don’t commit without an I\-D\-E.\-“ • Bezüglich Doxygen haben wir festgestellt, dass es ein comfortables und leistungsstarkes Tool zum automatischen Generieren von H\-T\-M\-L-\/\-Dokumentationen ist. • Doxygen unterstützt nur C Code, aber keine Präprozessor-\/\-Anweisungen. 